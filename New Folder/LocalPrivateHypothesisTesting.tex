\documentclass{article}
\usepackage{listings}
\usepackage{url}
\usepackage{mathtools}
\usepackage{amssymb}
\begin{document}
\title{Review of \emph{Local Private Hypothesis Testing:}\\ $\chi^2$ tests.}
\maketitle
\section{The BibTex}
First, this has the following BibTex:
\begin{lstlisting}
@ARTICLE{2017arXiv170907155G,
   author = {{Gaboardi}, M. and {Rogers}, R.},
    title = "{Local Private Hypothesis Testing: Chi-Square Tests}",
  journal = {ArXiv e-prints},
archivePrefix = "arXiv",
   eprint = {1709.07155},
 primaryClass = "math.ST",
 keywords = {Mathematics - Statistics Theory, Computer Science - Cryptography and Security},
     year = 2017,
    month = sep,
   adsurl = {http://adsabs.harvard.edu/abs/2017arXiv170907155G},
  adsnote = {Provided by the SAO/NASA Astrophysics Data System}
}
\end{lstlisting}
\section{The Paper}
We'll try to answer the following questions:
\begin{enumerate}
\item \emph{What new background is needed for this article}?

This uses the concept of \emph{concentrated differential privacy}.
This is a variant of differential privacy where the ``defining inequality'' doesn't hold for \emph{all events}.
Recall that for $(\epsilon,\delta)$ differential privacy, we say that for any two neighboring databases that:
\begin{equation}
\Pr[M(x) \in S]\leq e^{\epsilon}\Pr[M(y)\in S] + \delta
\end{equation}
must hold for \emph{all} subsets $S\subseteq\text{Range}(M(x))$.

Concentrated differential privacy instead says that:
\begin{equation}
\mathbb{E}_{y\sim\mathcal{M}(x)}[\exp(t\ln(\frac{\Pr[M(x) = y}{\Pr[M(y) = y}) - \rho)]\leq e^{t^2\rho},\quad \forall t\geq 0
\end{equation}
Note that the $\ln\frac{f(x)}{f(y)} - \rho$ quantity is \emph{essentially} the same thing as $\epsilon$ in the definition of $(\epsilon,\rho)$ differential privacy.
If $M(x)$ was $(\epsilon,\rho)$ differentially private, we'd have that this inequality is just (bounded by):
\begin{equation}
\mathbb{E}[\exp(t\epsilon)] \leq \exp(t^2\rho)
\end{equation}

We can view the Advanced Composition Theorem as a \emph{concentration inequality}, saying that \emph{privacy loss} is concentrated about the mean.

Concentrated Differential privacy makes a \emph{new} (incomparable) definition of differential privacy, with the intent that it satisfies a concentration inequality under composition.
The tradeoffs are that in $(\epsilon,\delta)$ DP, with probability $\delta$ privacy loss can be \emph{arbitrarily bad}.
Concentrated differential privacy essentially says that the \emph{privacy loss} is small mean, and sub-gaussian.


\item \emph{What new techniques do the authors apply?}


\item \emph{What new results do the authors get?}

They develop a $\chi^2$ test for \emph{local, concentrated} differential privacy.
This is the first of its kind (other concentrated differential privacy ones existed, 

\end{enumerate}
\end{document}